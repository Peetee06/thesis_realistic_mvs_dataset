\documentclass[a4paper,cleardoubleempty,BCOR1cm]{scrbook}
\input{header}

\title{Thesis Template}
\author{Peter Trost\thanks{e-mail: peter.trost@student.uni-tuebingen.de}}
\date{\today}
\begin{document}

\begin{tabular}{lr}
% \includegraphics[width=0.5\linewidth]{logo_sw} % logo bw
 \includegraphics[width=0.5\linewidth]{UT_WBMW_Rot_4C} % logo red
 & \hspace{0.2\linewidth}
 \parbox{0.5\linewidth}{
   \large\bf\textsf{\color{rot}{Mathematisch-\\Naturwissenschaftliche\\Fakultät\\\\}}
   \hspace{-.144cm}\normalsize\textsf{\color{rot}{Lernbasierte Computer Vision}}
   \vspace{0.6cm}
 }
\end{tabular}

\vspace*{10ex}
Masterarbeit

{\huge\bf\textsf{Title of Thesis}}

\vspace*{30ex}

Eberhard Karls Universität Tübingen\\
Mathematisch-Naturwissenschaftliche Fakultät\\
Wilhelm-Schickard-Institut für Informatik\\
Lernbasierte Computer Vision\\
My Name,~ \verb+my.name@uni-tuebingen.de+,~ 2012

\vspace*{5ex}

\begin{tabular}{@{}l@{\hspace{2em}}l}
  Bearbeitungszeitraum:& von-bis \vspace*{5ex} \\
  Betreuer/Gutachter:& Prof. Dr. Andreas Geiger, Universität Tübingen\\
  Zweitgutachter:& Prof. Dr. Max Mustermann, Universität Tübingen
\end{tabular}

\thispagestyle{empty}
\newpage

\chapter*{Selbstst\"andigkeitserkl\"arung}
Hiermit versichere ich, dass ich die vorliegende Masterarbeit selbst\"andig und
nur mit den angegebenen Hilfsmitteln angefertigt habe und dass alle Stellen,
die dem Wortlaut oder dem Sinne nach anderen Werken entnommen sind,
durch Angaben von Quellen als Entlehnung kenntlich gemacht worden sind.
Diese Masterarbeit wurde in gleicher oder \"ahnlicher Form in keinem anderen
Studiengang als Pr\"ufungsleistung vorgelegt.

\vspace*{8ex}
\hrule
\vspace*{2ex}
My Name (Matrikelnummer 123456), \today



\chapter*{Abstract}
Template

\chapter*{Acknowledgments}
If you have someone to Acknowledge ;)

\tableofcontents

%% braucht kein Mensch ...
%\listoffigures
%\listoftables

% write content here or...
\chapter{Introduction}
What is this all about?

Cite like this: \cite{agarwal2011}

\chapter{Related Work}
\section{Synthetically rendered datasets}
\subsection{A naturalistic open source movie for optical flow evaluation}
\cite{Butler:ECCV:2012}
\subsubsection{Overview}
In this paper the authors provide a dataset for optical flow estimation derived from the open source 3D animated short film Sintel
\todo{cite Sintel: https://durian.blender.org/}.
The dataset contains long sequences, large motions, specular reflections, motion blur, defocus blur, atmospheric effects and more. Its scenes are rendered in varying complexity through the source graphics data provided by the authors of the film. Because of this aforementioned variety the dataset can be used to improve optical flow methods. 

\subsubsection{Render passes}
As mentioned above the dataset contains image sequences rendered in the following variying complexity:\\
\begin{itemize}
	\item Albedo Pass: Flat and unshaded. Surfaces exhibit constant albedo over time
	\item Clean Pass: Illumination including smooth shading and specular reflections adds realism
	\item Final Pass: Full rendering with all effects including blur due to camera depth of field and motion, and atmospheric effects.
\end{itemize}

\subsubsection{Main aspects}
The main aspects of the Sintel dataset are the following:\\
It contains varying and more challenging (for existing methods) scenes than older datasets. Sequences are 50 frames long and are provided with 49 ground truth flow fields which are a measure of changes in position for objects in the scene from frame to frame. Some frames include motions of well over 100 pixels. There are 1628 frames with 564 for testing and 1064 for training. The Sintel dataset contains sequences having real-world challenges like lighting variations, shadows, complex materials, reflections and more.

\subsubsection{Meta}
The authors modified Blender's internal motion blur pipeline to give accurate motion vectors at each pixel which provide ground truth optical flow maps. Although the clips are selected so that optical flow is realistic, one still has to be cautios when training and evaluating algorithms that strongly rely on real-world laws of physics. The images are saved as 8-bit PNG files and the clips have a framerate of 24 fps.

\subsection{Playing for data: Ground truth from computer games}
\cite{Richter_2016_ECCV}

\subsection{The synthia dataset: A large collection of synthetic images for semantic segmentation of urban scenes}
\cite{RosCVPR16}

\subsection{SyB3R: A Realistic Synthetic Benchmark for 3D Reconstruction from Images}
\cite{syb3r2016}








\section{Problem Statement}
\todo{what you have to do here :)}

% ... input content via other .tex files
\chapter{Conclusion}
\label{sec:conclusion}

This work gave a foundation on domain adaptation and Generative Adversarial Networks, an overview over domain adaptation techniques and compared three of these techniques (namely CycleGAN \cite{DBLP:journals/corr/ZhuPIE17}, CyCADA \cite{DBLP:journals/corr/abs-1711-03213} and SG-GAN \cite{DBLP:journals/corr/abs-1801-01726}) on their capabilities to adapt synthetic images from the GTA dataset \cite{Richter_2016_ECCV} to real looking images from the Cityscapes dataset \cite{Cordts_2016_CVPR} domain. This was achieved by using pre-trained models for the techniques to translate a sample set of 500 images from the GTA images to the Cityscapes domain. A pre-trained DeepLabv3 \cite{DBLP:journals/corr/ChenPSA17} model was used to perform semantic segmentation on these generated images. The resulting predicted semantic segmentation images were then compared to the corresponding ground truth label maps of the GTA dataset using the code provided in the Cityscapes repository \cite{CSR}. This yields the Intersection over Union values for several categories and classes which the quantitative comparison is based on.
Of the three methods CycleGAN, SG-GAN and CyCADA only the latter one was able to improve semantic segmentation through adaptation from the synthetic GTA images to the real Cityscapes domain. 

\section{Outlook and Future Work}

The field of domain adapaption for semantic segmentation from synthetic to real domains is a very popular field of research especially regarding the many resources and research that currently go into autonomous driving. In order to make autonomous vehicles reliable and consistent the currently available domain adaptation techniques will still need more improvement. A possibility to improve upon this work could be to train own models for each of the compared techniques. This ensures a fair comparison. Another would be to incorporate more techniques and other domain adaptation tasks such as indoor synthetic to real adaptation. 

\appendix
\chapter{Blub}

\bibliographystyle{alpha}
\bibliography{bibliography}

\end{document}

