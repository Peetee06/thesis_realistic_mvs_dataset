\chapter{Foundations}
\label{sec:foundations}

This chapter will describe and explain the foundations necessary for this work. The term \textbf{Domain Adaptation} will be defined and described. Furthermore all relevant \textbf{Neural Network} architectures will be shown and explained. Specifically \textbf{Convolutional Neural Networks} and \textbf{Generative Adversarial Networks}.


\section{Domain Adaptation}
As described in \cite{DBLP:journals/corr/Csurka17}
Domain Adaptation is the task of transfering a machine learning model that is working well on a source data distribution to a related target data distribution. In this work we will focus on the adaptation from synthetic to real images. Synthetic meaning that the image was rendered from a virtual scene and real meaning an image taken from a real-world scene. In general there are many more domains, for example an image of a computer monitor from a front view (amazon) and one of a top view of the same monitor. \todo{insert example images}


\section{Neural Networks}

\subsection{Convolutional Neural Networks}
see \cite{wu2017introduction} (Introduction to CNNs)
Convolutional Neural Networks (CNNs) are Deep Neural Networks consisting of convolution layers, that extract features from input data (e.g. edges, curves, circles), pooling layers, commonly using max-pooling (mapping a subregion of the input to its maxium value) or average pooling (mapping the subregion to the average of the values) and a fully connected network for classification. 
\todo{add more description and example images}

\todo{convolution layer}

\todo{pooling layer}

\todo{fully connected layer}

\todo{example architectures VGG16, AlexNet,..?}


\subsection{Generative Adversarial Networks}
Generative Adversarial Networks (GANs) implement a two-player-game:\\
A Discriminator learns from a given data distribution what is ``real''. The Generator generates data. The goal of the generator is to fool the discriminator into believing the generated data is ``real''. The discriminator will label anything as ``fake'' that doesn't resemble the learned ``real'' data distribution. This way GANs can learn to generate realistically looking images of faces, translate images of art from one to another style and improve semantic segmentation.
The generative model generally uses \textit{maximum likelihood estimation}. Ian Goodfellow describes this in the following way in \cite{DBLP:journals/corr/Goodfellow17}:
\begin{quote}
	The basic idea of maximum likelihood is to define a model that provides an estimate of probability distribituion, parametereized by parameters $\theta$. We then refer to the \textbf{likelihood} as the probability that the model assigns to the training data: $\prod_{i=1}^{m}p_{\text{model}}(x^{(i)}; \theta)$
\end{quote}
The parameter $\theta$ that maximizes the likelihood of the data is better found in $\log$ space \cite{DBLP:journals/corr/Goodfellow17}
\begin{align}
	\theta^* &= \underset{\theta}{\arg \max} \prod_{i = 1}^{m} p_{\text{model}} (x^{(i)}; \theta)\\
	&= \underset{\theta}{\arg \max} \log \prod_{i=1}^{m} p_{\text{model}}(x^{(i)}; \theta)\\
	&= \underset{\theta}{\arg \max} \sum_{i = 1}^{m} \log p_{\text{model}}(x^{(i)}; \theta)
\end{align}
as the maximum of the function is at the same $\theta$ value and we now have a sum which aswell eliminates the possibility of having underflow by multiplying multiple very small probabilities together.

\todo{explain how discriminator works}

\todo{explain training}

\todo{explain advantages/disadvantages}