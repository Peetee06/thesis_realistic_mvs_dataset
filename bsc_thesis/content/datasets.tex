\chapter{Datasets}
\label{sec:datasets}

This chapter describes the \textbf{datasets} that are used during the \textbf{Comparison} of \textbf{Domain Adaptation Techniques}. Specifically it will show examples of the \textbf{Playing for Data} Grand Theft Auto V synthetic dataset, the \textbf{City-scapes} real dataset and \todo{add SUNCG and real dataset}.

\section{Playing for Data: Ground Truth from Computer Games}
see \cite{Richter_2016_ECCV}

\begin{itemize}
	\item contains 24966 images taken from a street view of Grand Theft Auto V (GTA5) in $1914 \times 1052$ pixels
	\item two orders of magnitude larger thatn CamVid and three orders of magnitude larger than semantic segmentation created for KITTI dataset
	\item highly realistic with moving cars, objects, pedestrians, bikes, day/night, changing lighting and weather conditions
	\item includes labels for these images
	\item labeling process took 49 hours, 3 magnitudes faster than comparable real datasets (normal annotation would've approximately taken 12 person-years)
	\item annotation took 7 seconds per image on average (514 times faster tahn for CamVid, 771 times faster than for Cityscapes)
	\item achieved by detouring: injecting a wrapper between game and graphics hardware to log functioncalls and reconstruct 3D scene
	\item objects in that scene can be assigned an object ID
	\item labeling an object in one image will then propagate that label to that object in every image that contains it
\end{itemize}
\todo{add some image samples}
\todo{add diversity of collected data graph}

\section{City-scapes}
see \cite{Cordts_2016_CVPR}

\begin{itemize}
	\item 30 classes
	\item 50 citites
	\item spring, summer, fall
	\item different weather conditions
	\item 5000 annotated images with fine annotations
	\item 20000 annotated images with coarse annotations
	\item 
\end{itemize}
