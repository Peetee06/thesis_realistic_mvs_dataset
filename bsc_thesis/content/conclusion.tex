\chapter{Conclusion}
\label{sec:conclusion}

This work gave a foundation on domain adaptation and Generative Adversarial Networks, an overview of domain adaptation techniques and compared three of these techniques (namely CycleGAN \cite{DBLP:journals/corr/ZhuPIE17}, CyCADA \cite{DBLP:journals/corr/abs-1711-03213} and SG-GAN \cite{DBLP:journals/corr/abs-1801-01726}) on their capabilities to adapt synthetic images from the GTA5 dataset \cite{Richter_2016_ECCV} to realistic-looking images from the Cityscapes dataset \cite{Cordts_2016_CVPR} domain. This was achieved by using pre-trained models for the techniques to translate a sample set of 500 images from the GTA5 images to the Cityscapes domain. A pre-trained DeepLabv3 \cite{DBLP:journals/corr/ChenPSA17} model was used to perform semantic segmentation on these generated images. The resulting predicted semantic segmentation images were then compared to the corresponding ground truth label maps of the GTA5 dataset using the code provided in the Cityscapes repository \cite{CSR}. This yields the Intersection over Union values for several categories and classes which the quantitative comparison is based on.
Of the three methods CycleGAN, SG-GAN and CyCADA only the latter was able to improve semantic segmentation through adaptation from the synthetic GTA5 images to the real Cityscapes domain.

\paragraph{}
Getting the different code to run proved difficult. First attempts were performed on the tcml cluster of the university of Tübingen \cite{tcml}. The cluster uses a management software to assign ressources and run code on the different nodes in the cluster. Images containing the environment used to run the different programs had to be built locally and then uploaded to the cluster. These images have filesizes of several hundred megabytes which made debugging very difficult. It became clear that this approach would not lead to results in a timely manner. Due to not having access to a machine with sufficient graphics hardware, the author tried to adapt the code of the different techniques to run on CPU. This worked for the DeepLabV3 model, Cityscapes evaluation code as well as for the SG-GAN model. However it did not work for the CycleGAN model and also took more than 12 hours for computing the translated images with the SG-GAN model. Luckily the author was able to find a machine with an NVIDIA Geforce GTX 1070 graphics card. Using this it took 2 days to set up the different environments for each technique and compute the translated images as well as evaluation thereof. 


\section{Outlook and Future Work}

The field of domain adapaption for semantic segmentation from synthetic to real domains is a very popular field of research especially regarding the many resources and research that currently go into autonomous driving. In order to make autonomous vehicles reliable and consistent the currently available domain adaptation techniques will still need more improvement. 

\paragraph{}
Even though the goal of this thesis was reached there are possibilities to improve upon and extend the comparison. One possibility to improve upon this work could be to train different models for each of the compared techniques. This way techniques can be compared on parameters like for example training set size and number of training iterations and epochs. Another would be to increase the size of the comparison set of images. Cornercases like the afore-mentioned ''train`` class could see more stable average results and might show different results to the ones obtained in this thesis. Additionaly other comparison metrics can be used. An example would be perceptual loss which compares the features contained in an image. To extend on this work it makes sense to incorporate more or different Domain Adaptation techniques. It would also be interesting to compare techniques on other domain adaptation tasks such as indoor synthetic to real adaptation. 