\chapter{Conclusion}
\label{sec:conclusion}

This work gave a foundation on domain adaptation and Generative Adversarial Networks, an overview over domain adaptation techniques and compared three of these techniques (namely CycleGAN \cite{DBLP:journals/corr/ZhuPIE17}, CyCADA \cite{DBLP:journals/corr/abs-1711-03213} and SG-GAN \cite{DBLP:journals/corr/abs-1801-01726}) on their capabilities to adapt synthetic images from the GTA dataset \cite{Richter_2016_ECCV} to real looking images from the Cityscapes dataset \cite{Cordts_2016_CVPR} domain. This was achieved by using pre-trained models for the techniques to translate a sample set of 500 images from the GTA images to the Cityscapes domain. A pre-trained DeepLabv3 \cite{DBLP:journals/corr/ChenPSA17} model was used to perform semantic segmentation on these generated images. The resulting predicted semantic segmentation images were then compared to the corresponding ground truth label maps of the GTA dataset using the code provided in the Cityscapes repository \cite{CSR}. This yields the Intersection over Union values for several categories and classes which the quantitative comparison is based on.
Of the three methods CycleGAN, SG-GAN and CyCADA only the latter one was able to improve semantic segmentation through adaptation from the synthetic GTA images to the real Cityscapes domain. 

\section{Outlook and Future Work}

The field of domain adapaption for semantic segmentation from synthetic to real domains is a very popular field of research especially regarding the many resources and research that currently go into autonomous driving. In order to make autonomous vehicles reliable and consistent the currently available domain adaptation techniques will still need more improvement. A possibility to improve upon this work could be to train own models for each of the compared techniques. This ensures a fair comparison. Another would be to incorporate more techniques and other domain adaptation tasks such as indoor synthetic to real adaptation. 