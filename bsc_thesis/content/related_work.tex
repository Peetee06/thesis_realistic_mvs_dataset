\chapter{Related Work}
\label{sec:related_work}
Following the initial GAN paper \cite{NIPS2014_5423}, many additions and alterations have been made to improve and stabilize the training process of GANs and creating better results. While some approaches use additional objectives like a conditional input to generate specific outputs \cite{DBLP:journals/corr/IsolaZZE16} on paired data, others use a cycle-consistency loss first proposed in \cite{DBLP:journals/corr/ZhuPIE17} making unpaired data possible aswell. This chapter shows some of these improved methods while first putting them into context of other methods of Domain Adaptation. 

\section{Approaches of Domain Adaptation}
Domain Adaptation can be subcategorized according to \cite{DBLP:journals/corr/Csurka17} and \cite{DBLP:journals/corr/abs-1802-03601}. The following sections specify these subcategories and will give example works each.

\subsection{Discrepancy based}
Discrepancy based techniques try to adapt domains by fine-tuning the deep network with labeled or unlabeled target data. Discrepancy based techniques can further be subdivided in following categories:
\subsubsection{Class Criterion}

\subsubsection{Statistic Criterion}

\subsubsection{Architecture Criterion}

\subsubsection{Geometric Criterion}

\subsection{Adversarial based}

\subsubsection{Generative models}
Also known as GANs. Section \ref{sec:DA_with_GANs} shows a few examples of interesting GAN techniques.

\subsubsection{Non-generative models}


\subsection{Reconstruction based}

\subsubsection{Encoder-Decoder Reconstruction}

\subsubsection{Adversarial Reconstruction}
Includes techniques that use a reconstruction error such as the cycle-consistency loss in CycleGAN (see Chapter \ref{sec:techniques} for more details).


%\subsection{Adapting Visual Category Models to New Domains}
%\cite{10.1007/978-3-642-15561-1_16}



\section{Domain Adaptation with Generative Adversarial Networks}
\label{sec:DA_with_GANs}
\subsection{CoGAN}
\cite{DBLP:journals/corr/0001T16}

\subsection{cGAN}
\cite{DBLP:journals/corr/IsolaZZE16}

\subsection{PixelDA}
\cite{DBLP:journals/corr/BousmalisSDEK16}