\chapter{Related Work}
\label{sec:related_work}

There are many approaches of adapting images from one domain to a different one. After the initial release of the GAN paper \cite{NIPS2014_5423} a lot of research focused on using and improving the GAN architecture and creating Benchmarks to evaluate the diverse approaches. This chapter will show some of these approaches.

\section{Classical Domain Adaptation Methods}


\section{Domain Adaptation with Generative Adversarial Networks}
Following the initial GAN paper \cite{NIPS2014_5423}, many additions and alterations have been made to improve and stabilize the training process and creating better results. While some approaches use additional objectives like a conditional input to generate specific outputs \cite{DBLP:journals/corr/IsolaZZE16} on paired data, others use a cycle-consistency loss first proposed in \cite{DBLP:journals/corr/ZhuPIE17} making unpaired data possible aswell. 

\todo{add info from ''Deep Visual Domain Adaptation - A Survey``-paper}

\subsection{Method 1}

\subsection{Method 2}

\subsection{Method 3}

\subsection{Method 4}

\subsection{Method 5}