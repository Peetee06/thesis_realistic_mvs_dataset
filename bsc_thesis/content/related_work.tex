\chapter{Related Work}
\label{sec:related_work}
Following the initial GAN paper \cite{NIPS2014_5423}, many additions and alterations have been made to improve and stabilize the training process of GANs and creating better results. While some approaches use additional objectives like a conditional input to generate specific outputs \cite{DBLP:journals/corr/IsolaZZE16} on paired data, others use a cycle-consistency loss first proposed in \cite{DBLP:journals/corr/ZhuPIE17} making unpaired data possible aswell. This chapter shows some of these improved methods and will also go over a few ``classical'' domain adaptation methods that don't use GANs.

\section{Classical Domain Adaptation Methods}

\subsection{Adapting Visual Category Models to New Domains}
\cite{10.1007/978-3-642-15561-1_16}



\section{Domain Adaptation with Generative Adversarial Networks}


\todo{add info from ''Deep Visual Domain Adaptation - A Survey``-paper}

\subsection{CoGAN}
\cite{DBLP:journals/corr/0001T16}

\subsection{cGAN}
\cite{DBLP:journals/corr/IsolaZZE16}

\subsection{PixelDA}
\cite{DBLP:journals/corr/BousmalisSDEK16}